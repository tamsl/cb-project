\documentclass[11pt]{article}
\usepackage[dutch]{babel}
\usepackage{hyperref}
\hypersetup{pdfborder={0 0 0 0}}
\author{Alexandra Moraga Pizarro (6129544)
      \\Tamara Ockhuijsen (6060374)
      \\Fredo Tan (6132421)}
\title{Compilerbouw
     \\Peephole optimizer}
\setlength{\parindent}{0pt}
\setlength{\parskip}{5pt plus 2pt minus 1pt}
\frenchspacing
\sloppy
\begin{document}
\maketitle
\newpage

\section{Inleiding}
Dit verslag beschrijft de uitwerking van de praktische opdracht voor het vak 
Compilerbouw, gegeven aan de Universiteit van Amsterdam in het jaar 2011/2012. 
Voor deze opdracht is een peephole optimizer voor SimpleScalar DLX assembly 
code ontwikkeld. Er zijn verschillende benchmark programma's in de taal C 
meegegeven. Een gcc cross compiler kan deze C-code compileren in assembly code 
die dient als invoer voor de peephole optimizer. Deze leest de assembly code 
en genereert nieuwe assembly code waarin overbodige instructies zijn 
verwijderd. De functionaliteit blijft hetzelfde als die van de originele 
assembly code, maar het programma werkt wel sneller.

\section{Implementatie}
De PLY package is gebruikt voor het parsen van assembly code. Dit is een 
ingebouwde implementatie van lex en yacc voor Python.

De implementatie is onderverdeeld in het parsen met lex en yacc en de 
optimalisatie.
\subsection{Lex}

\subsection{Yacc}

\subsection{Optimalisatie}
De volgende optimalisaties zijn toegepast op de assembly code.
\begin{verbatim}

Original sequence        Replacement

mov $regA,$regB     	   ---

mov $regA,$regB
instr $regA, $regA,...   instr $regA, $regB,...

instr $regA,...          instr $4,...
mov $4, $regA            jal XXX
jal  XXX

sw $regA,XXX             sw $regA, XXX
ld $regA,XXX

shift $regA,$regA,0      ---

add $regA,$regA,X        lw ...,X($regA)
lw ...,0($regA)

  beq ...,$Lx              bne ...,$Ly
  j $Ly                  $Lx:
$Lx: ...

\end{verbatim}

\section{Resultaten}

\section{Conclusie}

\section{Referenties}

\url{http://www.science.uva.nl/~andy/compiler/users_guide_v2.pdf}\\
\url{http://www.dabeaz.com/ply/}\\
\url{http://staff.science.uva.nl/~andy/compiler/prac.html}

\end{document}
